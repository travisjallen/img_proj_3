\documentclass[]{article}
\usepackage{graphicx}
\usepackage{amsmath,amssymb,amsthm}
\usepackage{empheq}
\usepackage{float}
\usepackage[left=0.85in,top=0.85in,right=0.85in,bottom=0.85in]{geometry} % Document margins

% Title Page
\title{CS 6640 Project 3}
\author{Travis Allen, u1056595}


\begin{document}
	\maketitle
	
	\newpage
	\section{Part 1: FFT's}
	\textbf{Use the built-in FFT functions in python (i.e., numpy) to compute the Fourier transform of some images and show their power spectrum.  You should organize the frequencies so that zero is in the middle of the FFT result/images.  You might need to use a log to see the power clearly.   Show results on several images and comment on what you see in both domains.  Implement a lower pass filter in the Fourier domain and do the inverse FFT to show the effects on several images.}

	\vskip 10pt

	Please see \texttt{problem\_1.py} for how I implemented this. 
	
	\vskip 10pt
	
	\subsection{Images and their FFTs}
	I used the \texttt{numpy.fft.fft2()} function to compute the discrete fast fourier transform of each of my images. Then, I used \texttt{numpy.fft.fftshift()} to ``retile" each image so that the center of each image corresponds to a frequency of zero. The only purpose of this step is to make the results (both intermediate and final) easier to digest. Then, I computed the log of the power spectrum of each image so the results were easier to see. These results are shown below for each image.
		
	%-------------------------------------------------------------------------------------------
	% image 0
	\begin{figure}[H]
		\centering
		\includegraphics[width=6.5in]{p1_output/img_0_ft_compare.png}
		\caption{Image 0 and its fourier transform}
	\end{figure}
	
%	%-------------------------------------------------------------------------------------------
%	% image 1
%	\begin{figure}[H]
%		\centering
%		\includegraphics[width=6.5in]{p1_output_keep/img_1_t_nl1.png}
%		\caption{Image 1, Noise level = 1}
%	\end{figure}
%	
%	%-------------------------------------------------------------------------------------------
%	% image 2
%	\begin{figure}[H]
%		\centering
%		\includegraphics[width=6.5in]{p1_output_keep/img_2_t_nl1.png}
%		\caption{Image 2, Noise level = 1}
%	\end{figure}
%	
%	
%	%-------------------------------------------------------------------------------------------
%	% image 3
%	\begin{figure}[H]
%		\centering
%		\includegraphics[width=6.5in]{p1_output_keep/img_3_t_nl1.png}
%		\caption{Image 3, Noise level = 1}
%	\end{figure}
%	
%	\newpage
%	
	\subsection{Low Pass Filtering in the Fourier Domain}
		Typically we think of low pass filtering in the spatial domain as convolution of the image with some function. From the properties of the fourier transform we know that convolution in the spatial domain is the same as multiplication in the fourier domain. This means that we can multiply the fourier transform of the image with the fourier transform of our filtering function, and then take the inverse  fourier transform of that product to achieve the same result as convolution in the spatial domain. 
	
	I implemented this in two ways. First, I created a 2D rect$(u,v)$ function in the fourier domain (sinc$(x,y)$). This is shown below:
	
	\begin{figure}[H]
		\centering
		\includegraphics[width=3.75in]{p1_output/rect_filter.png}
		\caption{rect$(u,v)$ filter in the Fourier Domain}
	\end{figure}
	
	
	I element-wise multiplied each pixel in the filter with each pixel in the image. We know that although this is a mathematically ``ideal" function, in practice it produces artifacts that we are not interested in, like ringing. 
	
	% image 1
	\begin{figure}[H]
		\centering
		\includegraphics[width=5.25in]{p1_output/img_0_f_rect.png}
		\caption{Original image and fourier-domain rect$(u,v)$ filtered image. Note that this is the same as convolution with a sinc$(x,y)$ function in the spatial domain. This filter appears to darken the image as well. This image has a lot going on in it, so it is difficult to notice the ringing.}
	\end{figure}

	\begin{figure}[H]
		\centering
		\includegraphics[width=6.5in]{p1_output/img_1_f_rect.png}
		\caption{Original image and fourier-domain rect$(u,v)$ filtered image. Notice the subtle ringing above the harsh transition between the ceramic surrounding the electromagnet and the background in the upper third of the image. This is an unwanted artifact of the rect$(u,v)$ filter.}
	\end{figure}
	
	
	Next, I created a 2D Gaussian distribution function in the fourier domain. We know that $\mathcal{F} \{\text{exp}(-\pi u^2)\} = \text{exp}(-\pi x^2)$, i.e. the fourier transform of a gaussian is a gaussian, so it was not necessary that I create this filter in the fourier domain so long as the images in the fourier domain are retiled (shifted) to be filtered properly with this mask. To make the filter, I multiplied a gaussian in $x$ with a gaussian in $y$:
	%---------------------------------------------------------------------------------
	% gaussian
	\begin{figure}[H]
		\centering
		\includegraphics[width=5.5in]{p1_output/gaussian_filter.png}
		\label{enumeratred_mse}
	\end{figure}

The results of this filter are shown below:
	
	\begin{figure}[H]
		\centering
		\includegraphics[width=6.5in]{p1_output/img_0_f_gaussian.png}
		\caption{Original image and fourier-domian, gaussian low pass filtered image. Notice the slight blurring in the filtered image.}
	\end{figure}

\newpage
	\section{Part 2: Phase Correlation}
	Please see \texttt{problem\_2.py} for my implementation of this part. Please note that to run the code you will either need to have \texttt{numba} or you will need to comment the decorator at line \texttt{106}. 
	
	First, I implemented raw phase correlation with no low pass filtering. This works according to the following equation:
	\[p(x, y)=\mathbb{F}^{-1}\left[\frac{F^{*}(u, v) G(u, v)}{\left|F^{*}(u, v) G(u, v)\right|}\right]\]
	
	%---------------------------------------------------------------------------------
	% image 0 bilateral
	\begin{figure}[H]
		\centering
		\includegraphics[width=6.5in]{p2_output/phase_correlation_raw.png}
		\caption{Phase correlation of images 0 and 1. The raw correlation data is shown on the left, and a thresholded version is shown on the right. The thresholded version is provided because it is much easier to interpret. See the lower right corner for the pixel of highest intensity.}
	\end{figure}
	\newpage
Next, I implemented a low pass filter on the above phase correlation before I took the inverse fourier transform. I used the same gaussian low pass filter from part 1 for this. Curiously, I think these results are worse.
	\begin{figure}[H]
		\centering
		\includegraphics[width=6.5in]{p2_output/phase_correlation_low_pass.png}
		\caption{Phase correlation of images 0 and 1. The low pass filtered correlation data is shown on the left, and a thresholded version is shown on the right. The thresholded version is provided because it is much easier to interpret. it is slightly harder to see, but there is a white pixel at the location of the highest intensity in the lower right corner.}
	\end{figure}

\newpage
\section{Peak Finding}	
\subsection{Algorithm Development - Na\"ive Implementation}
\label{alg_dev_peak}
We want to compute the phase correlation of two overlapping images and find the relative location of one with respect to the other so that when we overlay them they line up. Without loss of generality, we will call these images image 1 and image 2.
\begin{figure}[H]
	\centering
	\includegraphics[width=6.75in]{images/region.png}
	\label{regions}
	\caption{Possible locations of image 1 with respect to image 2, and the regions in image 2 defined by these relative locations.}
\end{figure}
Suppose image 1 is an array of $mn$ elements, where $m$ is the number of rows and $n$ is the number of columns. We can only compute the phase correlation of two images if they are of the same size, so it follows that image 2 is an $m$x$n$ array as well. If we compute the phase correlation of images 1 and 2, we have an image where the value of each element is the phase correlation at that location. We can find the indices of the maximal element in the array and call this location $(\lambda _x, \lambda _y)$. Note that in this case, the axis corresponding to the second dimension is defined as positive when we traverse \emph{down} the image, contrary to our intuition from Cartesian axes. We are using python, so as far as the computer is concerned we can say that $(\lambda _x, \lambda _y)\equiv$ \texttt{image2}$[\lambda _y,\lambda _x]$.

\vskip 5pt

\textbf{Region 0:} Consider the placement of image 1 on top of image 2 that defines region 0. In this case, we wish to place the origin of image 1 (the upper left corner) a directed distance of $(-{(n-\lambda _x)},-{(m-\lambda _y)})$ from the origin of image 2. This means that region 0 in image 1 is defined as the rectangle whose vertices are located at $(n-\lambda _x,m-\lambda _y)$, $(n,m-\lambda _y)$, $(n-\lambda_x,m)$, and $(n,m)$. We want to compute the phase correlation of this region with the same region in image 2, whose vertices are located at $(0,0)$, $(\lambda_x,0)$, $(0,\lambda_y)$, and $(\lambda_x,\lambda_y)$. Once we have computed the phase correlation of these two regions, we can record the maximum intensity of the phase correlation.

\vskip 5pt

\textbf{Region 1:} Consider the placement of image 1 on top of image 2 that defines region 1. In this case, we wish to place the origin of image 1 a directed distance of $(\lambda_x,-{(m-\lambda_y)})$ away from the origin of image 2. Thus, the vertices of region 1 in image 1 are given by $(0,m-\lambda_y)$, $(n-\lambda_x,m-\lambda_y)$, $(0,\lambda_y)$, and $(n-\lambda_x,m)$. The vertices of region 1 in image 2 are $(n-\lambda_x,0)$, $(n,0)$, $(n-\lambda_x, \lambda_y)$, and $(n-\lambda_x,m)$. We then compute the phase correlation of these two regions and record the maximum intensity of the phase correlation.

\vskip 5pt

\textbf{Region 2:} Consider the placement of image 1 on top of image 2 that defines region 2. In this case, we wish to place the origin of image 1 a directed distance of $(-(n-\lambda_x),\lambda_y)$ away from the origin of image 2. Thus, the vertices of region 2 in image 1 are given by $(n-\lambda_x,0)$, $(n,0)$, $(n-\lambda_x,m-\lambda_y)$, and $(n,m-\lambda_y)$. The vertices of region 2 in image 2 are $(0,\lambda_y)$, $(\lambda_x,\lambda_y)$, $(0,m)$, and $(\lambda_x,m)$. Once again, we compute the phase correlation of these two regions and record the maximum intensity.

\vskip 5pt

\textbf{Region 3:} Finally, consider the placement of image 1 on top of image 2 that defines region 3. In this case, we wish to place the origin of image 1 a directed distance of $(\lambda_x,\lambda_y)$ away from the origin of image 2. Thus, the vertices of region 3 in image 1 are given by $(0,0)$, $(n-\lambda_x,0)$, $(0,m-\lambda_y)$, and $(n-\lambda_x,m-\lambda_y)$. The vertices of region 3 in image 2 are $(\lambda_x,\lambda_y)$, $(n,\lambda_y)$, $(\lambda_x,m)$, and $(n,m)$. For the last time we compute the phase correlation of these two regions and record the maximum intensity.

Now we have four maximum intensities, one coming from each region. We expect three of them to be meaningless (though not necessarily of the same value or order of magnitude) and one to be the maximal among the four. The region which contributes this maximal intensity dictates the appropriate placement of image 1 on top of image 2. 

Now we must define a canvas on which to place the images so we can visually inspect the results of this approach. We can easily determine the size of the canvas based on our new knowledge of the correct location of image 1 with respect to image 2. If we end up with region 0, the canvas must have $(2n-\lambda_x)$ columns and $(2m-\lambda_y)$ rows. If we end up with region 1, the canvas must have $(n+\lambda_x)$ columns and $(2m-\lambda_y)$ rows. If we end up with region 2, the canvas must have $(2n-\lambda_x)$ columns and $(m+\lambda_y)$ rows. Finally, if we end up with region 3, the canvas must have $(n+\lambda_x)$ columns and $(m+\lambda_y)$ rows. 

\newpage
\subsection{Results - Na\"ive Implementation}
Using the above algorithm I was able to take the following pairs of images:
\begin{figure}[H]
	\centering
	\includegraphics[width=6.5in]{p3_output/img_0_pieces.png}
\end{figure}

\begin{figure}[H]
	\centering
	\includegraphics[width=6.5in]{p3_output/img_1_pieces.png}
\end{figure}

\begin{figure}[H]
	\centering
	\includegraphics[width=6.5in]{p3_output/img_3_pieces.png}
\end{figure}

And stitch them together into the following mosaics:

\begin{figure}[H]
	\centering
	\includegraphics[width=6.5in]{p3_output/img_0_mosaic.png}
\end{figure}

\begin{figure}[H]
	\centering
	\includegraphics[width=6.5in]{p3_output/img_1_mosaic.png}
\end{figure}

\begin{figure}[H]
	\centering
	\includegraphics[width=6.5in]{p3_output/img_3_mosaic.png}
\end{figure}

\vskip 5pt
I have called this the ``na\"ive implementation" because I am simply finding the maximum phase correlation and locating purely based on that metric. This has no built in low-pass filtering method (Gaussian, Butterworth, finding centroids of connected components, or some other clever method) and could be susceptible to unforseen difficulties relating to high frequency noise. However, this is unlikely because we can think of high frequency noise as being related to the number of sharp edges in an image. Both of these images are composed of many edges, meaning many high frequency signals, so, if this method really were so delicate, these particular images should expose this weakness. However, looking at the results we can see that these reconstructions are nearly perfect, indicating that this is a valid way of constructing mosaics that appears to be immune to high frequency noise.
\newpage
\subsection{Introducing a Low Pass Filter}
Once this was working, I multiplied the shifted phase correlation in the fourier domain with the gaussian low pass filter described and shown in part 1 of the project, prior to taking the inverse shift and inverse fourier transform of the result. Curiously, this seemed to only make my results worse. The most noticeable case was with the last image I showed. Without the low pass filter, the reconstruction was almost completely perfect: every fine line in each image lined up as far as the eye could tell. When I implemented the low pass filter, I got the following results:

\begin{figure}[H]
	\centering
	\includegraphics[width=6.5in]{p3_output/img_3_mosaic_lp.png}
	\caption{Now the images are slightly offset from each other, which is most clearly seen in the illustration of the broad leaved nira right where it meets the other image.}
\end{figure}

\newpage
I also tried this method on the image of the electromagnet:

\begin{figure}[H]
	\centering
	\includegraphics[width=6.5in]{p3_output/img_1_mosaic_lp.png}
	\caption{Similarly, the results are worse than without the gaussian low pass filter. The details are a little more subtle here, but there is a definite horizontal line where the two images meet.}
\end{figure}


\newpage

\section{Part 4: Mosaic Building}
\subsection{Algorithm Development}
We want to build a mosaic out of an arbitrary number of images. We know that some of these images overlap, but we don't know how many overlap, which particular images overlap with the others, and how many sides of each image overlap. How do we even begin?

First, I arbitrarily place the first image in the center of a large canvas. In this case, and in all cases, I consider the origin of the image to be \texttt{image[0,0]} or the upper left corner of the current image. Then, I iterate through the remaining images to find an image that overlaps with the one I have just plotted. I check the overlap by comparing the maximal element of the phase correlation with the average value of the phase correlation. If the ratio of the average over the maximum is less than a certain value, that means that the maximum phase correlation is large, indicating that there is overlap. In practice, the cutoff ratio requires tuning, and I arrived at an arbitrary value of 6.2e-5, over which the two images did not overlap.

Now that we know that the images overlap, it's just a matter of placing the images in the correct location. The method in section \ref{alg_dev_peak} tells us the directed distance away from the origin of one image we need to place the origin of the other. In the case of doing this repeatedly, the reference image is the last image we plotted and the next image is just the next image in the queue. We continue to shift the location of the origin that we care about in a manner that matches with figure 11.

\begin{figure}[H]
	\centering
	\label{coord_sys}
	\includegraphics[width=5in]{coord_sys.png}
	\caption{A possible configuration of overlapping images, showing the coordinate systems with respect to which each pair was overlapped.}
\end{figure}

If we encounter an image that does not overlap, we simply skip it for the time being and continue through the queue of images. Once we reach the end of the queue, assuming not all images have been plotted, we work backwards toward the front of the queue following the same procedure. This process repeats until we have plotted all of the images. 

\subsection{Results}
This method works fairly robustly. Shown below are some results with the cell images and one of the images from part 1.
\begin{figure}[H]
	\centering
	\includegraphics[width=6.5in]{images/cell_images_subplot.png}
	\caption{Cell images pre-mosaicing. For the most part it's hard to tell how they would line up.}
\end{figure}

\newpage
Here they are matched:
\begin{figure}[H]
	\centering
	\includegraphics[width=5.5in]{cell_complete_up_close.png}
	\caption{These images line up well, but there is a slight mismatch on the bottom two images. This is likely because they are the first and last images placed, and they are not referenced to each other, only to the next and previous images in the chain.}
\end{figure}

\begin{figure}[H]
	\centering
	\includegraphics[width=6.5in]{lola_complete_small_hole.png}
	\caption{These images also line up. The missing patch in the middle is due to my very imprecise cropping procedure. Careful attention to the parts of the image that aren't around the hole shows that the images line up.}
\end{figure}

\subsection{Instructions for Implementation}
To implement this script on your own computer, you must do the following:
\begin{itemize}
	\item Place all of the images in a folder with a known path to the directory that contains \texttt{problem\_4.py}
	\item Place all of the names of all of the images in a \texttt{.txt} file in the folder, with each name separated by a new line
	\item Write the names of the folder and the file in lines \texttt{27} and \texttt{28} of \texttt{problem\_4.py}
	\item Write the maximum size of the images in lines \texttt{21} and \texttt{22} of \texttt{problem\_4.py}
	
\end{itemize}



\end{document}          

